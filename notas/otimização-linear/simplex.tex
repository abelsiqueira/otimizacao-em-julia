\documentclass{article}
\usepackage[utf8]{inputenc}
\usepackage[T1]{fontenc}
\usepackage{amsthm,amsmath}

\usepackage{algorithm,algpseudocode}

\author{Abel Soares Siqueira}
\title{Método Simplex - Descrição Básica}
\date{Última atualização 2021/Fev/15 \\ CC-BY-SA}

\begin{document}
  \maketitle

  \begin{algorithm}
    \caption{Simplex (uma iteração)}
    \begin{algorithmic}[1]
      \State Dado $x$ solução básica factível, com base $B$ e não-base $N$.
      % \State Defina $k = 0$
      \State Calcule $y$ resolvendo $B^T y = c_B$.
      \State Calcule $\bar{c} = c - A^T y$.
      \State Se $\bar{c} \geq 0$, TERMINE com $x^* = x$ solução ótima.
      \State Escolha $j$ tal que $\bar{c}_j < 0$.
      \State Calcule $d$ tal que
        \begin{align*}
          d_j & = 1 \\
          B d_B & = A_j \\
          d_i & = 0, \quad \forall i \in N-\{j\}.
        \end{align*}
      \State Se $d \geq 0$, TERMINE com o problema ilimitado na direção $d$.
      \State Calcule $\displaystyle k = \arg\min_{i: d_i < 0} \frac{-x_i}{d_i}$.
      \State Defina $\theta = \dfrac{-x_k}{d_k}$.
      \State Calcule $x = x + \theta d$, que equivale a
      \begin{align*}
        x_B & \leftarrow x_B + \theta d_B \\
        x_k & \leftarrow 0 \\
        x_j & \leftarrow \theta \\
        x_i & \quad \text{continua valendo } 0, \quad \forall i \in N-\{j\}.
      \end{align*}
      \State Remova $k$ de $B$ e adicione em $N$.
      \State Remova $j$ de $N$ e adicione em $B$.
    \end{algorithmic}
  \end{algorithm}

\end{document}